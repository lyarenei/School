\documentclass[12pt]{article}
\usepackage[utf8]{inputenc}
\usepackage[margin=1in]{geometry}
\usepackage{amsfonts, amsmath, amssymb}
\usepackage{fancyhdr}
\usepackage{graphicx}
\usepackage{float}
\usepackage[free-standing-units]{siunitx}
\usepackage[european]{circuitikz}
\usepackage{tikz}
\usepackage{tabularx}
\usepackage{tabu}

\pagestyle{fancy}
\fancyhead{}

\begin{document}

\begin{titlepage}
\begin{center}
\vspace*{2cm}
\Large{\textbf{Elektrotechnika pro informační technologie}}\\
\textbf{Semestrální práce}\\

\vfill
\includegraphics[scale=0.125]{pics/fit_logo.png}
\vfill

\textbf{Dominik Křivohlávek (xkrivo02)}\\
\textbf{1BIA 2016}\\

\end{center}
\end{titlepage}

\fancyhead[L]{První úloha}
\fancyhead[R]{\includegraphics[scale=0.025]{pics/fit_logo.png}}
\section{První úloha (varianta: A)}
\subsection{Zadání}
\begin{center}

\normalsize
Stanovte napětí $U_{R8}$ a proud $I_{R8}$. Použijte metodu postupného zjednodušování obvodu.\\
\vspace{15px}

$U_{1} = 80$ V$, U_{2} = 120$ V$, R_{1} = 350$  $\Omega, R_{2} = 650$  $\Omega, R_{3} = 41$  $\Omega, R_{4} = 130$  $\Omega, R_{5} = 360$  $\Omega, R_{6} = 750$  $\Omega, R_{7} = 310$  $\Omega, R_{8} = 190$ $\Omega.$\\
\vspace{15px}

\begin{circuitikz} \draw
 (0,0) to (1,0)
 (1,0) to (1,-3)
 (1,0) to[R=$R_1$, *-] (4,0)
 (1,-3) to[R=$R_2$] (4,-3)
 (4,-3) to[R=$R_3$, *-*] (4,0)
 (4,0) to[R=$R_4$] (7,0)
 (4,-3) to[R=$R_5$,-*] (7,-3)
 (7,0) to (7,-5)
 (7,-5) to[R=$R_6$,-*] (4,-5)
 (4,-5) to (4,-4)
 (4,-5) to (4,-6)
 (4,-4) to[R=$R_7$,] (1,-4)
 (4,-6) to[R=$R_8$, i>^=$I_{R_{8}}$, v=$U_{R_{8}}$, -*] (1,-6)
 (1,-4) to (1,-6)
 (1,-6) to (0,-6)
 (0,-6) to (0,-5)
 (0,-5) to[dcvsource, v<=$U_2$] (0,-3)
 (0,-3) to[dcvsource, v<=$U_1$, i^=$I$] (0,0)
;\end{circuitikz}

\end{center}
\vspace{25px}

\subsection{Řešení}

\begin{center}

\normalsize
Zjednodušíme zapojení $R_{7}$ a $R_{8}$. Zároveň přepočítáme trojúhelníkové zapojení $R_{1}$, $R_{2}$ a $R_{3}$ na hvězdu:\\
\vspace{15px}

\begin{circuitikz} \draw
 (1,0) to[R=$R_a$, -*] (4,0)
 (4,0) to (4,1)
 (4,1) to[R=$R_b$] (7,1)
 (7,1) to[R=$R_4$] (10,1)
 (10,1) to (10,-3)
 (4,0) to (4,-1) 
 (4,-1) to[R=$R_c$] (7,-1)
 (7,-1) to[R=$R_5$, -*] (10,-1)
 (10,-3) to[R=$R_6$] (7,-3)
 (7,-3) to[R=$R_{78}$] (4,-3)
 (4,-3) to[dcvsource, v<=$U_2$] (1,-3)
 (1,-3) to[dcvsource, v<=$U_1$, i^=$I$] (1,0)
;\end{circuitikz}
\vspace{25px}

\Large
$R_{a} = \frac{R_{1}*R_{2}}{R_{1}+R_{2}+R_{3}} = \frac{350*650}{350+650+410} = 161,3475$  $\Omega$\\
\vspace{10px}

$R_{b} = \frac{R_{1}*R_{3}}{R_{1}+R_{2}+R_{3}} = \frac{350*410}{350+650+410} = 101,773$  $\Omega$\\
\vspace{10px}

$R_{c} = \frac{R_{2}*R_{3}}{R_{1}+R_{2}+R_{3}} = \frac{650*410}{350+650+410} = 189,007$  $\Omega$\\
\vspace{10px}

$R_{78} = \frac{R_{7}*R_{8}}{R_{7}+R_{8}} = \frac{310*190}{310+190} = 117,8$  $\Omega$\\
\vspace{25px}

\normalsize
Dále zjednodušíme zapojení rezistorů $R_{b}$ s $R_{4}$, $R_{c}$ s $R_{5}$ a $R_{78}$ s $R_{6}$:
\vspace{25px}

\begin{circuitikz} \draw
 (1,0) to[R=$R_a$, -*] (4,0)
 (4,0) to (4,1)
 (4,1) to[R=$R_{b4}$] (7,1)
 (7,1) to (7,-3)
 (4,0) to (4,-1) 
 (4,-1) to[R=$R_{c5}$, -*] (7,-1)
 (7,-3) to[R=$R_{678}$] (4,-3)
 (4,-3) to[dcvsource, v<=$U_2$] (1,-3)
 (1,-3) to[dcvsource, v<=$U_1$, i^=$I$] (1,0)
;\end{circuitikz}
\vspace{25px}

\Large
$R_{b4} = R_{b}+R_{4} = 101,773+130 = 231,773$  $\Omega$\\
\vspace{10px}

$R_{c5} = R_{c}+R_{5} = 189,007+360 = 549,007$  $\Omega$\\
\vspace{10px}

$R_{678} = R_{6}+R_{78} = 117,8+750 = 867,8$  $\Omega$\\
\clearpage

\normalsize
Následně zjednodušíme paralelní zapojení $R_{b4}$ s $R_{c5}$:
\vspace{25px}

\begin{circuitikz} \draw
 (1,0) to[R=$R_a$] (4,0)
 (4,0) to[R=$R_{b4c5}$] (7,0)
 (7,0) to (7,-3)
 (7,-3) to[R=$R_{678}$] (4,-3)
 (4,-3) to[dcvsource, v<=$U_2$] (1,-3)
 (1,-3) to[dcvsource, v<=$U_1$, i^=$I$] (1,0)
;\end{circuitikz}
\vspace{25px}

\Large
$R_{b4c5} = \frac{R_{b4}*R_{c5}}{R_{b4}+R_{c5}} = \frac{231,773*549,007}{231,773+549,007} = 162,9716$  $\Omega$\\
\vspace{15px}

\normalsize
Na závěr spočítáme výsledný celkový odpor obvodu:
\vspace{15px}

\begin{circuitikz} \draw
 (0,0) to[dcvsource, v<=$U$, i^=$I$] (2,0)
 (2,0) to[R=$R$] (4,0)
 (4,0) to (4,-1)
 (4,-1) to (0,-1)
 (0,-1) to (0,0)
;\end{circuitikz}
\vspace{15px}

\Large
$R = R_{a}+R_{b4c5}+R_{678} = 161,3475+162,9716+867,8 =$\\$ = 1192,1191$  $\Omega$\\
\vspace{25px}

\normalsize
Jelikož nyní známe celkový odpor obvodu, můžeme spočítat celkový proud procházející obvodem:
\vspace{15px}

\Large
$I = \frac{U_{1}+U_{2}}{R} = \frac{120+80}{1192,1191} = 167,7768$ mA
\vspace{20px}

\normalsize
Nyní můžeme spočítat napětí na rezistoru $R_{78}$. Jelikož je napětí na paralelních větvích shodné, bude toto napětí i na rezistoru $R_{8}$:\\
\vspace{15px}

\Large
$U_{78} = U_{R8} = R_{78}*I = 117,8*0,1677 = 19,755$ V
\vspace{20px}

\normalsize
Nakonec spočítáme ze známého napětí na $R_{8}$ proud jím procházející:
\vspace{15px}

\Large
$I_{R8} = \frac{U_{R8}}{R_{8}} = \frac{19,755}{190} = 103,974$ mA

\end{center}
\clearpage

\fancyhead[L]{Druhá úloha}
\fancyhead[R]{\includegraphics[scale=0.025]{pics/fit_logo.png}}
\section{Druhá úloha (varianta: F)}
\subsection{Zadání}
\begin{center}

\normalsize
Stanovte napětí $U_{R4}$ a proud $I_{R4}$. Použijte metodu Théveninovy věty.\\
\vspace{15px}

$U = 130$ V, $R_{1} = 350$  $\Omega, R_{2} = 600$  $\Omega, R_{3} = 195$  $\Omega, R_{4} = 650$  $\Omega, R_{5} = 280$  $\Omega.$\\
\vspace{15px}

\begin{circuitikz} \draw
 (0,0) to[R=$R_1$, -*] (3,0)
 (3,0) to[R=$R_2$, *-*] (3,-3)
 (3,0) to (6,0)
 (6,0) to[R=$R_3$, *-*] (6,-3)
 (6,0) to (9,0)
 (9,0) to[R=$R_4$, v>=$U_{R_{4}}$, i>^=$I_{R_{4}}$] (9,-3)
 (9,-3) to (6,-3)
 (6,-3) to[R=$R_5$, *-*] (3,-3)
 (3,-3) to (0,-3) 
 (0,-3) to[dcvsource, v<=$U$, i^=$I$] (0,0)
;\end{circuitikz}

\end{center}
\vspace{15px}

\subsection{Řešení}
\begin{center}

\normalsize
Pro spočítání vnitřního odporu $R_{i}$ odpojíme z obvodu zkoumaný rezistor $R_{4}$ a zkratujeme zdroj napětí U:\\
\vspace{15px}

\begin{circuitikz} \draw
 (3,0) to[R=$R_1$, -*] (0,0)
 (3,0) to[R=$R_2$, *-*] (3,-3)
 (3,0) to (6,0)
 (6,0) to[R=$R_3$] (6,-3)
 (3,-3) to[R=$R_5$] (6,-3)
 (3,-3) to (0,-3) 
 (0,-3) to (0,0)
;\end{circuitikz}
\vspace{15px}

Dále můžeme spočítat sériové zapojení rezistorů $R_{3}$ a $R_{5}$:\\
\vspace{15px}

\begin{circuitikz} \draw
 (3,0) to[R=$R_1$, -*] (0,0)
 (3,0) to[R=$R_2$, *-*] (3,-3)
 (3,0) to (6,0)
 (6,0) to[R=$R_{35}$] (6,-3)
 (6,-3) to (3,-3)
 (3,-3) to (0,-3) 
 (0,-3) to (0,0)
;\end{circuitikz}
\vspace{15px}

\Large
$R_{35} = R_{3}+R_{5} = 195+280 = 475$  $\Omega$\\
\vspace{25px}

\normalsize
Následně spočítáme odpor paralelně zapojených rezistorů $R_{2}$ a $R_{35}$:\\
\vspace{15px}

\begin{circuitikz} \draw
 (3,0) to[R=$R_1$] (0,0)
 (3,0) to[R=$R_{235}$] (3,-3)
 (3,-3) to (0,-3) 
 (0,-3) to (0,0)
;\end{circuitikz}
\vspace{15px}

\Large
$R_{235} = \frac{R_{2}*R_{35}}{R_{2}+R_{35}} = \frac{475*600}{475+600} = 265,1162$  $\Omega$\\
\vspace{25px}

\normalsize
Z předchozího obrázku je zřejmé sériové zapojení rezistorů $R_{1}$ a $R_{235}$. Spočítáme jejich společný odpor pro získání $R_{i}$:\\
\vspace{15px}

\begin{circuitikz} \draw
 (0,0) to[R=$R_i$] (3,0)
 (3,0) to (3,-1)
 (3,-1) to (0,-1) 
 (0,-1) to (0,0)
;\end{circuitikz}
\vspace{15px}

\Large
$R_{i} = R_{1}+R_{235} = 265,1162+350 = 615,1162$  $\Omega$\\
\vspace{25px}

\normalsize
Se znalostí celkového vnitřního odporu $R_{i}$ můžeme spočítat celkový proud v obvodu:\\
\vspace{15px}

\Large
$I = \frac{U}{R_{i}} = \frac{130}{615,1162} = 0,2113$ A\\
\vspace{25px}

\normalsize
Nyní můžeme spočítat napětí na rezistoru $R_{235}$:\\
\vspace{15px}

\Large
$U_{R_{235}} = R_{i}*I = 615,1162*0,2113 = 56,019$ V\\
\vspace{25px}

\normalsize
Následně spočítáme proud procházející větví s rezistory $R_{i}$ a $R_{i}$:\\
\vspace{15px}

\Large
$I_{R_{35}} = \frac{U_{R_{235}}}{R_{35}} = \frac{56,019}{475} = 0,1179$ A\\
\clearpage

\normalsize
Se znalostí $I_{R_{35}}$ můžeme spočítat napětí na rezistoru $R_{3}$. Jelikož je na paralelních větvích obvodu stejné napětí, bude se toto napětí rovnat vnitřnímu napětí $U_{i}$:\\
\vspace{15px}

\Large
$U_{R_{3}} = U_{i} = R_{3}*I_{R_{35}} = 195*0,1179 = 22,9905$ V\\
\vspace{15px}

\normalsize
Nyní známe potřebné veličiny pro náhradní obvod s vnitřním napětím $U_{i}$ a vnitřním odporem $R_{i}$. Připojíme tedy $R_{4}$ a spočítáme proud jím procházející. Také spočítáme napětí na rezistoru $R_{4}$:\\
\vspace{15px}

\begin{circuitikz} \draw
 (0,0) to[R=$R_i$] (3,0)
 (3,0) to[R=$R_4$, v>=$U_{R_{4}}$, i>^=$I_{R_{4}}$] (3,-3)
 (3,-3) to (0,-3) 
 (0,-3) to[dcvsource, v<=$U_i$] (0,0)
;\end{circuitikz}
\vspace{15px}

\Large
$I_{R_{4}} = \frac{U_{i}}{R_{i}+R_{4}} = \frac{22,9905}{615,1162+650} = 18,1726$ mA\\
\vspace{25px}

$U_{R_{4}} = R_{4}*I_{R_{4}} = 650*0,1817 = 11,8$ V\\

\end{center}
\clearpage

\fancyhead[L]{Třetí úloha}
\fancyhead[R]{\includegraphics[scale=0.025]{pics/fit_logo.png}}
\section{Třetí úloha (varianta: E)}
\subsection{Zadání}
\begin{center}

\normalsize
Stanovte napětí $U_{R_{4}}$ a proud $I_{R_{4}}$. Použijte metodu uzlových napětí ($U_A$, $U_B$, $U_C$).\\
\vspace{15px}

$U = 135$ V, $I_1 = 0.55$ A, $I_2 = 0.65$ A$, R_{1} = 52$  $\Omega, R_{2} = 42$  $\Omega, R_{3} = 52$  $\Omega, R_{4} = 42$  $\Omega,$\\$R_{5} = 21$  $\Omega$\\
\vspace{25px}

\begin{circuitikz}[scale=1.4] \draw
 (0,0) to (1,0)
 
 (1,-3) to[R=$R_1$, v_<=$U_A$, *-*] (1,0)
 (1,0) to[R=$R_2$] (4,0)
 (4,0) to[open, v^=$U_B$] (1,-3) 
 (4,0) to[R=$R_3$, *-*] (4,-3)
 (4,0) to[R=$R_5$] (7,0)
 (7,0) to[dcvsource, v>=$U$] (7,-3)
 (7,-3) to (4,-3)
 (4,-3) to[R=$R_4$, v>=$U_C$, i>^=$I_{R_{4}}$, *-*] (1,-3)
 (4,-3) to (4,-4)
 (4,-4) to[ioosource, i=$I_1$] (1,-4)
 (1,-4) to (1,-3)
 (1,-3) to (0,-3) 
 (0,-3) to[ioosource, i^=$I_2$] (0,0)
 
 (3.5,0.0005) to[open, l=2](4.5,0.0005)
 (0.5,0.0005) to[open, l=1](1.5,0.0005)
 (3.5,-3.65) to[open, l=3](5,-3.65)
 (-0.5,-3.65) to[open, l=0](2,-3.65)
 
;\end{circuitikz}
\clearpage
\end{center}

\subsection{Řešení}
\begin{center}

\normalsize
Přepočítáme rezistory na vodivosti a také zdroj napětí na zdroj proudu:\\
\vspace{15px}

\Large
$G_1=G_3=\frac{1}{R_1}=\frac{1}{52}=0.0192$ S\\
\vspace{10px}

$G_2=G_4=\frac{1}{R_2}=\frac{1}{42}=0.0238$ S\\
\vspace{10px}

$G_5=\frac{1}{R_5}=\frac{1}{21}=0.0476$ S\\
\vspace{10px}

$I_i=\frac{U}{R_5}=\frac{135}{21}=6.4285$ A\\
\vspace{25px}

\begin{circuitikz}[scale=1.4] \draw
 (0,0) to (1,0)
 (1,-3) to[R=$G_1$, v_<=$U_A$, *-*] (1,0)
 (1,0) to[R=$G_2$] (4,0)
 (4,0) to[open, v^=$U_B$] (1,-3) 
 (4,0) to[R=$G_3$, *-*] (4,-3)
 (4,0) to[R=$G_5$] (7,0)
 (7,0) to[ioosource, i<=$I_i$] (7,-3)
 (7,-3) to (4,-3)
 (4,-3) to[R=$G_4$, v>=$U_C$, i>^=$I_{R_{4}}$, *-*] (1,-3)
 (4,-3) to (4,-4)
 (4,-4) to[ioosource, i=$I_1$] (1,-4)
 (1,-4) to (1,-3)
 (1,-3) to (0,-3) 
 (0,-3) to[ioosource, i^=$I_2$] (0,0)
 
 (3.5,0.0005) to[open, l=2](4.5,0.0005)
 (0.5,0.0005) to[open, l=1](1.5,0.0005)
 (3.5,-3.65) to[open, l=3](5,-3.65)
 (-0.5,-3.65) to[open, l=0](2,-3.65)
 
;\end{circuitikz}
\vspace{25px}

\normalsize
Sestavíme matice pro výpočet:
\vspace{15px}

$\begin{bmatrix}
G_1+G_2 & -G_2 & 0\\
-G_2 & G_2+G_3+G_5 & -(G_3+G_5)\\
0 & -(G_3+G_5) & G_2+G_3+G_5\\
\end{bmatrix}
*
\begin{bmatrix}
U_A\\
U_B\\
U_C\\
\end{bmatrix}
=
\begin{bmatrix}
I_2\\
I_i\\
-I_1-I_i\\
\end{bmatrix}$
\vspace{15px}

$\begin{bmatrix}
0.043 & -0.0238 & 0\\
-0.0238 & 0.0906 & -0.0668\\
0 & -0.0668 & 0.0906\\
\end{bmatrix}
*
\begin{bmatrix}
U_A\\
U_B\\
U_C\\
\end{bmatrix}
=
\begin{bmatrix}
0.65\\
6.4285\\
-6.9785\\
\end{bmatrix}$
\clearpage

Spočítáme determinant matice:\\
\vspace{10px}$\Delta =[0.043*0.0906*0.0906]-[(-0.0238*(-0.0238)*0.0906)+(0.043*(-0.0668)*(-0.0668)]$\\
\vspace{10px}

$\Delta =0.000109763$\\
\vspace{15px}

Následně upravíme původní matici pro výpočet $U_C$ tak, že místo hodnot ze třetího sloupce matice dosadíme hodnoty z matice proudů a vypočítáme její determinant:
\vspace{15px}

$\Delta_C\begin{bmatrix}
G_1+G_2 & -G_2 & I_2\\
-G_2 & G_2+G_3+G_5 & I_i\\
0 & -(G_3+G_5) & -I_1-I_i\\
\end{bmatrix}$\\
\vspace{10px}

$\Delta_C\begin{bmatrix}
0.043 & -0.0238 & 0.65\\
-0.0238 & 0.0906 & 6.4285\\
0 & -0.0668 & -6.9785\\
\end{bmatrix}$
\vspace{10px}

$\Delta_C =[(0.043*0.0906*(-6.9785))+((-0.0238)*(-0.0668)*(0.65)]-[(-0.0238*(-0.0238)*(-6.9785))+(6.4285*(-0.668)*0.043)]$\\
\vspace{10px}

$\Delta_C=-0.003735319$\\
\vspace{10px}

\normalsize
Nyní podělíme determinanty mezi sebou, abychom získali napětí $U_C$. Jelikož je toto napětí mezi uzlem 3 a 0, kde je taktéž pouze rezistor $R_4$, jsou napětí $U_C$ a $U_{R_4}$ shodná:\\
\vspace{15px}

\Large
$U_C=U_{R_4}=\frac{\Delta_C}{\Delta}=\frac{-0.003735319}{0.000109763}=-34.0307$ V\\
\vspace{15px}

\normalsize
Z napětí na rezistoru poté snadno spočítáme proud $I_{R_4}$:\\
\vspace{15px}

\Large
$I_{R_4}=\frac{U_{R_4}}{R_4}=\frac{-34.0307}{42}=-0.8102$ A\\

\end{center}
\clearpage

\fancyhead[L]{Čtvrtá úloha}
\fancyhead[R]{\includegraphics[scale=0.025]{pics/fit_logo.png}}
\section{Čtvrtá úloha (varianta: A)}
\subsection{Zadání}
\begin{center}

\normalsize
Pro napájecí napětí platí: $u_1 = U_1*\sin(2\pi ft)$, $u_2 = U_2*\sin(2\pi ft)$. Ve vztahu pro napětí $u_{C_1} = U_{C_1}*\sin(2\pi ft+\varphi_{C_1})$ určete $|{U_{C_1}}|$ a $\varphi_{C_1}$. Použijte metodu smyčkových proudů.
\vspace{15px}

$U = 35$ V, $U_2 = 55$ V$, R_{1} = 12$  $\Omega, R_{2} = 14$  $\Omega, R_{3} = 10$  $\Omega, L_{1} = 120$ mH, $L_{2} = 21$ mH, $C_1 = 200$ $\mu$F, $C_2 = 105$ $\mu$F, $f = 70$ Hz\\
\vspace{25px}

\begin{circuitikz}[scale=1.4] \draw
(0,0) to[american inductor=$L_1$, *-] (2,0)
(2,0) to[R=$R_2$, -*] (4,0)
(4,0) to[american inductor=$L_2$, -*] (6,0)
(4,0) to[C=$C_2$, -*] (4,-2)
(6,0) to[R=$R_4$] (6,-2)
(6,-2) to[sV, v<=$U_2$] (4,-2)
(0,-2) to[C=$C_1$, v<=$u_{C_1}$, i^<=$i_{C_1}$] (4,-2)
(0,-2) to[sV, v<=$U_1$] (0,0)
(0,0) to (0,2)
(0,2) to[R=$R_1$] (6,2)
(6,2) to (6,0)
;\end{circuitikz}

\end{center}
\subsection{Řešení}
\begin{center}

\normalsize
Nejprve si spočítáme úhlovou rychlost $\omega$ a také reaktance cívek a kondenzátorů:\\
\vspace{15px}

\Large
$\omega = 2\pi f= 2*3.14*70=439.8229$ rad/s\\
\vspace{10px}

$X_{L_1}=j\omega L_1=j439.8229*120*10^{-3}=j52.7787$ $\Omega$\\
\vspace{10px}

$X_{L_2}=j\omega L_2=j439.8229*100*10^{-3}=j52.7787$ $\Omega$\\
\vspace{10px}

$X_{C_1}=-j\frac{1}{\omega C_1}=-j\frac{1}{200*10^{-6}}=-j11.3682$ $\Omega$\\
\vspace{10px}

$X_{C_2}=-j\frac{1}{\omega C_2}=-j\frac{1}{105*10^{-6}}=-j21.6537$ $\Omega$\\
\clearpage

\normalsize
Dále si v obvodu zvolíme smyčkové proudy a sestavíme matice pro jejich výpočet:
\vspace{15px}

\begin{circuitikz}[scale=1.4] \draw
(0,0) to[american inductor=$L_1$] (2,0)
(2,0) to[R=$R_2$, -*] (4,0)
(4,0) to[american inductor=$L_2$, -*] (6,0)
(4,0) to[C=$C_2$, -*] (4,-2)
(6,0) to[R=$R_4$] (6,-2)
(6,-2) to[sV, v<=$U_2$] (4,-2)
(0,-2) to[C=$C_1$, v<=$u_{C_1}$, i^<=$i_{C_1}$] (4,-2)
(0,-2) to[sV, v<=$U_1$] (0,0)
(0,0) to (0,2)
(0,2) to[R=$R_1$] (6,2)
(6,2) to (6,0)

(2,1) node[scale=3]{$\circlearrowright$}
(2,1) node{$I_C$}
(2.85,-1) node[scale=3]{$\circlearrowright$}
(2.85,-1) node{$I_A$}
(5.25,-1) node[scale=3]{$\circlearrowright$}
(5.25,-1) node{$I_B$}
;\end{circuitikz}
\vspace{15px}

\normalsize
$\begin{bmatrix}
X_{L_1}+R_2+X_{C_2}+X_{C_1} & -X_{C_2} & -(X_{L_1}+R_2)\\
-X_{C_2} & R_3+X_{C_2}+X_{L_1} & -X_{L_2}\\
-(X_{L_1}+R_2) & -X_{L_2} & R_1+X_{L_2}+R_2+X_{L_1}\\
\end{bmatrix}
*
\begin{bmatrix}
I_A\\
I_B\\
I_C\\
\end{bmatrix}
=
\begin{bmatrix}
-U_1\\
-U_2\\
0\\
\end{bmatrix}$
\vspace{15px}

\normalsize
$\begin{bmatrix}
14+j19.7568 & j21.6537 & -14-j52.7787\\
j21.6537 & 10+j22.3285 & -j43.9822\\
-14-j52.7787 & -j43.9822 & 26+j96.7609\\
\end{bmatrix}
*
\begin{bmatrix}
I_A\\
I_B\\
I_C\\
\end{bmatrix}
=
\begin{bmatrix}
-35\\
-55\\
0\\
\end{bmatrix}$
\vspace{15px}

\normalsize
Vypočítáme determinant:
\vspace{10px}

$\Delta\begin{bmatrix}
14+j19.7568 & j21.6537 & -14-j52.7787\\
j21.6537 & 10+j22.3285 & -j43.9822\\
-14-j52.7787 & -j43.9822 & 26+j96.7609\\
\end{bmatrix}$
\vspace{10px}

$\Delta=d_1-d_2$\\
\vspace{10px}

$d_1=[(14+j19.7568)*(10+j22.3285)*(26+j96.7609)]+[(j21.6537)*(-j43.9822)*(-14-j52.7787)]+[(-14-j52.7787)*(j21.6537)*(-j43.9822)]$\\
\vspace{10px}

$d_2=[(-14-j52.7787)*(10+j22.3285)*(-14-j52.7787)]+[(j21.6537)*(j21.6537)*(26+j96.7609)]+[(14+j19.7568)*(-j43.9822)*(-j43.9822)]$\\
\vspace{15px}

$\Delta=14305.6583+j10226.7032$\\
\vspace{10px}
\clearpage

Upravíme matici pro výpočet $I_A$:
\vspace{15px}

\normalsize
$A\begin{bmatrix}
-U_1 & -X_{C_2} & -(X_{L_1}+R_2)\\
-U_2 & R_3+X_{C_2}+X_{L_1} & -X_{L_2}\\
0 & -X_{L_2} & R_1+X_{L_1}+R_2+X_{L_1}\\
\end{bmatrix}$
\vspace{10px}

\large
Po dosazení:
\vspace{15px}

\normalsize
$\A\begin{bmatrix}
-35 & j21.6537 & -14-j52.7787\\
-55 & 10+j22.3285 & -j43.9822\\
0 & -j43.9822 & 26+j96.7609\\
\end{bmatrix}$
\vspace{15px}

Opět vypočítáme determinant:\\
\vspace{10px}

$\Delta_A=d_1-d_2$\\
\vspace{15px}

$d_1=[(-35)*(10+j22.3285)*(26+j96.7609)]+[(-55)*(-j43.9822)*(-14-j52.7787)]$\\
\vspace{10px}

$d_2=[(j21.6537)*(-55)*(26+j96.7609)]+[(-j43.9822)*(-j43.9822)*(-35)]$\\
\vspace{10px}

$\Delta_A=14305.6583+j10226.7032$\\
\vspace{15px}

Podělíme determinanty mezi sebou pro výpočet $I_A$. Jelikož se kondenzátor $C_1$ nachází na větvi, která nesousedí s jinými smyčkami, bude se smyčkový proud $I_A$ rovnat proudu $i_{C_1}$:\\
\vspace{15px}

\Large
$I_A=i_{C_1}=\frac{\Delta_A}{\Delta}=\frac{11248.2564-j57086.753}{14305.6583+j10226.7032}=-1.3675-j3.0128$ A
\vspace{15px}

\normalsize
Jelikož známe proud, můžeme vypočítat napětí $U_{C_1}$ na příslušném kondenzátoru:
\vspace{15px}

\large
$U_{C_1}=X_{C_1}*I_A=-j11.3682*-1.3675-j3.0128=-34.251+j15.5465$ V
\vspace{15px}

\normalsize
Spočítáme velikost napětí na kondenzátoru $C_1$:
\vspace{15px}

\large
$|U_{C_1}|=\sqrt{(-34.251)^2+(15.5465)^2}=37.6141$ V\\
\vspace{15px}

\normalsize
Nakonec vypočítáme fázový posun mezi reálnou a imaginární složkou napětí na kondenzátoru $C_1$:
\vspace{15px}

\Large
$\varphi_{U_{C_1}}=\tan^{-1}\frac{U_{C_{1_{im}}}}{U_{C_{1_r}}}=\tan^{-1}\frac{15.5465}{-34.251}=\tan^{-1} (-0.4538)=-24.4132\degree$\\
\vspace{15px}

\normalsize
A ještě upravíme: \large$\varphi_{U_{C_1}}=360\degree-24.4132\degree=335.5868\degree$



\end{center}
\clearpage

\fancyhead[L]{Pátá úloha}
\fancyhead[R]{\includegraphics[scale=0.025]{pics/fit_logo.png}}
\section{Pátá úloha (varianta: F)}
\subsection{Zadání}
\begin{center}
Sestavte diferenciální rovnici popisující chování obvodu na obrázku, dále ji upravte dosazením hodnot parametrů. Vypočítejte analytické řešení $i_L = f(t)$. Proveďte kontrolu výpočtu dosazením do sestavené diferenciální rovnice.\\
\vspace{10px}

$U = 45$ V, $L = 30$ H, $R = 15$ $\Omega$, $i_L(0) = 4$ A
\vspace{15px}

\begin{circuitikz}[scale=1.4] \draw
(0,0) to[R=$R$] (2,0)
(2,0) to[american inductor=$L$, i>^=$i_L$] (2,-2)
(0,-2) to[sV, v<=$U$] (0,0)
(0,-2) to (2,-2)
;\end{circuitikz}
\vspace{15px}

\end{center}
\subsection{Řešení}
\begin{center}
Při řešení využijeme axiómu pro proud $i'_L$ a též využijeme druhého Kirchhoffova zákona.
\vspace{10px}

\Large$i'_L = \frac{1}{L}*u_L$\\
\vspace{10px}

\normalsize Sestavíme rovnici dle II. KZ: $R*i_L + u_L - U = 0$, Vyjádříme $u_L$ a dosadíme do axiómu:\\
\vspace{10px}

\Large $i'_L = \frac{1}{L}*(U-R*i_L)$
\vspace{10px}

\normalsize Dosadíme hodnoty ze zadání (kromě $i_L$):
\vspace{10px}

\Large $i'_L = \frac{1}{30}*(45-15i_L)$
\vspace{10px}

\normalsize Úpravami získáme rovnici ve tvaru $y' + y = konstanta$:\\
\vspace{10px}

\Large $i'_L + \frac{1}{2}i_L = \frac{3}{2}$\\
\vspace{10px}

\normalsize Z rovnice použijeme koeficienty proměnných pro určení parametru $\lambda$:
\vspace{10px}

\Large $\lambda + \frac{1}{2} = 0$ ; $\lambda = -\frac{1}{2}$\\
\vspace{10px}

\normalsize Očekávaný stav $i_L$ popisuje rovnice $i_L(t) = c(t) * e^\lambda$. Dosadíme:\\
\vspace{10px}

\Large $i_L(t) = c(t) * e^{-\frac{1}{2}(t)}$
\vspace{10px}

\normalsize Rovnici zderivujeme pro získání $i'_L$:\\
\vspace{10px}

\Large $i'_L(t) = c(t)' * e^{-\frac{1}{2}(t)} - \frac{1}{2} c(t) * e^{-\frac{1}{2}(t)}$\\
\vspace{10px}

\normalsize Následně dosadíme $i'_L$ a $i_L$ do rovnice, ze které jsme určili parametr $\lambda$:
\vspace{10px}

\Large $i'_L + \frac{1}{2}i_L = \frac{3}{2}$\\
\vspace{10px}

$c(t)' * e^{-\frac{1}{2}(t)} - \frac{1}{2} c(t) * e^{-\frac{1}{2}(t)} + - \frac{1}{2} c(t) * e^{-\frac{1}{2}(t)} = \frac{3}{2}$\\
\vspace{10px}

$c(t)' * e^{-\frac{1}{2}(t)} = \frac{3}{2} \Rightarrow c(t)' = \frac{\frac{3}{2}}{e^{-\frac{1}{2}(t)}} \Rightarrow c(t) = \int \frac{\frac{3}{2}}{e^{-\frac{1}{2}(t)}}$\\
\vspace{10px}

\normalsize Vyřešíme integrál:\\
\vspace{10px}

\Large
$\int \frac{\frac{3}{2}}{e^{-\frac{1}{2}(t)}} = \frac{3}{2} * \frac{1}{2} * 2 * e^{\frac{1}{2}(t)} + c = \frac{3}{2} * e^{\frac{1}{2}(t)} + c$\\
\vspace{10px}

\normalsize Zintegrovaný výsledek dosadíme do původní rovnice, kterou jsme dříve derivovali:
\vspace{10px}

\Large
$i_L(t) = (\frac{3}{2} * e^{\frac{1}{2}(t)} + c) * e^{-\frac{1}{2}(t)}$
\vspace{10px}

\normalsize Ze zadání víme, že se $i_L$ při čase t = 0, rovná 4 A. Dosadíme tedy tuto hodnotu pro vypočtení konstanty c, která vznikla při integrování:
\vspace{10px}

\Large
$4 = (\frac{3}{2} * e^{\frac{1}{2}(0)} + c) * e^{-\frac{1}{2}(0)}$
\vspace{10px}

$4 = \frac{3}{2} + c \Rightarrow c = \frac{5}{2}$
\vspace{10px}

\normalsize Následně dosadíme zpětně do původní rovnice pro získání analytického řešení $i_L$:
\vspace{10px}

\Large $i_L(t) = (\frac{3}{2} * e^{\frac{1}{2}(t)} + \frac{5}{2}) * e^{-\frac{1}{2}(t)}$
\vspace{10px}

\normalsize Na závěr provedeme kontrolu výpočtem. Ze zadání víme, že $i_L(0) = 4$ A. Ověříme:\\
\vspace{10px}

\Large $i_L(0) = (\frac{3}{2} * e^{\frac{1}{2}(0)} + \frac{5}{2}) * e^{-\frac{1}{2}(0)}$
\vspace{10px}

$i_L(0) = \frac{3}{2} + \frac{5}{2} = 4$ A
\end{center}
\clearpage

\fancyhead[L]{Výsledky}
\fancyhead[R]{\includegraphics[scale=0.025]{pics/fit_logo.png}}
\section{Výsledky}
\vspace{30px}
\begin{center}

{\tabulinesep=1.2mm
\begin{tabu} {|c|c|}\hline
Zadání & Výsledky\\ \hline
1A & $U_{R8} = 19,755$ V, $I_{R8} = 103,974$ mA\\
\hline
2E & $U_{R_{4}} = 11,8$ V, $I_{R_{4}} = 18,1726$ mA\\
\hline
3F & $U_{R_4} = -34.0307$ V, $I_{R_4} = -0.8102$\\
\hline
4A & $|U_{C_1}| = 37.6141$ V, $\varphi_{U_{C_1}} = -24.4132\degree$\\
\hline
5F & $i_L(t) = (\frac{3}{2} * e^{\frac{1}{2}(t)} + \frac{5}{2}) * e^{-\frac{1}{2}(t)}$, $i_L(0) = 4$ A\\ \hline       
\end{tabu}}
\end{center}
\end{document}
