\documentclass[11pt,twocolumn, a4paper]{article}
\usepackage{times}
\usepackage[left=1.5cm, text={18cm, 25cm}, top=2.5cm]{geometry}
\usepackage[utf8]{inputenc}
\usepackage[czech]{babel}
\usepackage{amsmath}
\usepackage{amsfonts}
\usepackage{amsthm}
%
\theoremstyle{definition}
\newtheorem{definition}{Definice}[section]
\newtheorem{algorithm}[definition]{Algoritmus}
\newtheorem{theorem}{Věta}
%
\begin{document}
%
\begin{titlepage}
%
\begin{center}
%
\Huge
\textsc{Fakulta informačních technologií
Vysoké učení technické v Brně\\}
%
\vspace{\stretch{0.382}}
%
{\LARGE
Typografie a publikování - 2. projekt\\
Sazba dokumentů a matematických výrazů\\}
%
\vspace{\stretch{0.618}}
%
\end{center}
{\Large 2017 \hfill Dominik Křivohlávek}

\end{titlepage}

\section*{Úvod}
V této úloze si vyzkoušíme sazbu titulní strany, matematických vzorců, prostředí a dalších textových struktur obvyklých pro technicky zaměřené texty, například rovnice (1) nebo definice 1.1 na straně 1.\par
%
Na titulní straně je využito sázení nadpisu podle optického středu s využitím zlatého řezu. Tento postup byl probírán na přednášce.\par
%
\section{Matematický text}
Nejprve se podíváme na sázení matematických symbolů a výrazů v plynulém textu. Pro množinu $V$ označuje card$(V)$ kardinalitu $V$. Pro množinu $V$ reprezentuje $V^*$ volný monoid generovaný množinou $V$ s operací konkatenace. Prvek identity ve volném monoidu $V^*$ značíme symbolem $\varepsilon$. Nechť $V^+=V^*-\{\varepsilon\}$. Algebraicky je tedy $V^+$ volná pologrupa generovaná množinou $V$ s operací konkatenace. Konečnou neprázdnou množinu $V$ nazvěme \textit{abeceda}. Pro $w \in V^*$ označuje $|w|$ délku řetězce $w$. Pro $W \subseteq V$ označuje occur$(w,V)$ počet výskytů symbolů\linebreak z $W$ v řetězci $w$ a sym$(w,i)$ určuje $i$-tý symbol řetězce $w$; například sym$(abcd,3)=c$.\par
%
Nyní zkusíme sazbu definic a vět s využitím balíku {\ttfamily amsthm}.
\par
%
\begin{definition} \textit{Bezkontextová gramatika} je čtveřice $G = (V,T,P,S)$, kde $V$ je totální abeceda, $T \subseteq V$ je abeceda terminálů, $S \in (V - T)$ je startující symbol a $P$ je konečná množina \textit{pravidel} tvaru $q: A \rightarrow \alpha$, kde $A \in (V - T)$, $\alpha \in V^*$ a $q$ je návěští tohoto pravidla. Nechť $N = V - T$ značí abecedu neterminálů. Pokud $q: A \rightarrow \alpha \in P$ , $\gamma,\delta \in V^*$ , $G$ provádí derivační krok z $\gamma A \delta$ do $\gamma\alpha\delta$ podle pravidla $q\!: A \rightarrow \alpha$, symbolicky píšeme $\gamma A \delta \Rightarrow \gamma \alpha \delta\; [q: A \to \alpha]$ nebo zjednodušeně $\gamma A \delta \Rightarrow \gamma\alpha\delta$ . Standardním způsobem definujeme $\Rightarrow^m$, kde $m \geq 0$ . Dále definujeme tranzitivní uzávěr $\Rightarrow^+$ a tranzitivně-reflexivní uzávěr $\Rightarrow^*$ .\end{definition}
%
Algoritmus můžeme uvádět podobně jako definice textově, nebo využít pseudokódu vysázeného ve vhodném prostředí (například {\ttfamily algorithm2e}).\par
%
\begin{algorithm} \textit{Algoritmus pro ověření bezkontextovosti gramatiky. Mějme gramatiku G = (N, T, P, S)}.\end{algorithm}
%
\begin{enumerate}
\label{alg1} \item \textit{Pro každé pravidlo $p \in P$ proveď test, zda p na levé straně obsahuje právě jeden symbol z $N$}.
\item \textit{Pokud všechna pravidla splňují podmínku z kroku \ref{alg1}, tak je gramatika $G$ bezkontextová.}\par
\end{enumerate}
%
\begin{definition} \textit{Jazyk} definovaný gramatikou $G$ definujeme jako $L(G) = \{w \in T^* | S \Rightarrow^* w \}$ . \end{definition}
%
\subsection{Podsekce obsahující větu}
\begin{definition} Nechť $L$ je libovolný jazyk. $L$ je \textit{bezkontextový jazyk}, když a jen když $L = L(G)$, kde $G$ je libovolná bezkontextová gramatika. \end{definition}
%
\begin{definition} Množinu $\mathcal{L}_{CF} = \{L|L$ je bezkontextový jazyk$\}$ nazýváme \textit{třídou bezkontextových jazyků}. \end{definition}
%
\label{veta} \begin{theorem} \textit{Nechť} $L_{abc} = \{a^n b^n c^n |n \geq 0 \}$. Platí, že $L_{abc} \notin \mathcal{L}_{CF}$. \end{theorem}
%
\begin{proof}
Důkaz se provede pomocí Pumping lemma pro bezkontextové jazyky, kdy ukážeme, že není možné, aby platilo, což bude implikovat pravdivost věty \pageref{veta}.\end{proof}
%
\section{Rovnice a důkazy}
Složitější matematické formulace sázíme mimo plynulý text. Lze umístit několik výrazů na jeden řádek, ale pak je třeba tyto vhodně oddělit, například příkazem {\ttfamily \verb|\quad|}. 
%
$$\sqrt[x^2]{y_{0}^{3}} \quad \mathbb{N} = \{0,1,2,...\} \quad x^{y^y} \neq y^{yy} \quad z_{i_j} \not\equiv z_{ij}$$
%
\indent V rovnici (\ref{rovnice}) jsou využity tři typy závorek s různou explicitně definovanou velikostí.
%
\begin{eqnarray}
\label{rovnice} \bigg\{\Big[\big(a+b\big)*c\Big]^d +1\bigg\} & = & x\\
\lim_{x \to \infty} \frac{sin^2 x + cos^2 x}{4} & = & y \nonumber
\end{eqnarray}
%
\indent V této větě vidíme, jak vypadá implicitní vysázení limity $\lim_{n \to \infty} f(n)$ v normálním odstavci textu. Podobně je to i s dalšími symboly jako $\sum_{1}^{n}$ či $\bigcup_{A \in \mathcal{B}}$. V případě vzorce $\lim\limits_{x \to \infty} \frac{\sin x}{x} = 1$ jsme si vynutili méně úspornou sazbu příkazem {\ttfamily \verb|\limits|}. \par
%
\setcounter{equation}{1}
\begin{eqnarray}
\int_a^b f(x), \mathrm{d}x & = & -\int_b^a f(x), \mathrm{d}x\\ 
\left(\sqrt[5]{x^4}\right)' = \left(x^{\frac{4}{5}}\right)' & = & \frac{4}{5}x^{-\frac{4}{5}} = \frac{4}{5\sqrt[5]{x}}\\
\overline{\overline{A \lor B}} & = & \overline{\overline{A} \land \overline{B}}
\end{eqnarray}
%
\section{Matice}
%
Pro sázení matic se velmi často používá prostředí {\ttfamily array} a závorky ({\ttfamily \verb|\left|,\verb|\right|}). \par
%
$$ \left(
\begin{array}{c c}
a+b & b-a \\
\widehat{\xi + \omega} & \hat{\pi}\\
\vec{a} & \overleftrightarrow{AC}\\
0 & \beta\\
\end{array} \right)$$
%
$$A= \left| \left|
\begin{array}{c c c c} 
a_{11} & a_{12} & \cdots & a_{1n}\\
a_{21} & a_{22} & \cdots & a_{2n}\\
\vdots  & \vdots  & \ddots & \vdots\\
a_{m1} & a_{m2} & \cdots & a_{mn}\\
\end{array} \right| \right|$$
%
$$ \left| \begin{array}{c c}
t & u\\
v & w\\
\end{array} \right| = tw - uv$$
%
Prostředí {\ttfamily array} lze úspěšně využít i jinde.
%
$$\binom{k}{n} = 
\left\{ \begin{array}{l l}
\frac{n!}{k!(n-k)} & \text{pro } 0 \leq k \leq n \\
0 & \text{pro } k < 0 \text{ nebo } k > n
\end{array} \right. $$\par
%
\section{Závěrem}
V případě, že budete potřebovat vyjádřit matematickou konstrukci nebo symbol a nebude se Vám dařit jej nalézt v samotném \LaTeX u, doporučuji prostudovat možnosti balíku maker \AmS -\LaTeX .
Analogická poučka platí obecně pro jakoukoli konstrukci v \TeX u.
\end{document}