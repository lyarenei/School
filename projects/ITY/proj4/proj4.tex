\documentclass[11pt, a4paper]{article}
\usepackage{times}
\usepackage[left=2cm, text={17cm, 24cm}, top=3cm]{geometry}
\usepackage[utf8]{inputenc}
\usepackage[czech]{babel}
%
\bibliographystyle{czplain}
%
\begin{document}
%
\begin{titlepage}
%
\begin{center}
%
\Huge
\textsc{Vysoké učení technické v Brně\\
\huge Fakulta informačních technologií\\}
%
\vspace{\stretch{0.382}}
%
{\LARGE
Typografie a publikování -- 4. projekt\\
\Huge Bibliografické citace\\}
%
\vspace{\stretch{0.618}}
%
\end{center}
{\Large \today \hfill Dominik Křivohlávek}
\medskip
\end{titlepage}
%
\section{Úvod}
Bezdrátová technologie je již dnes součástí našeho života, a již těžce si lze život bez ní představit.
Mezi bezdrátovými technologiemi je nejspíše nejznámější WiFi \cite{wifi_pruvodce}. Pojďme se na tuto technologii podívat trochu podrobněji.
%
\section{Standardy}
%
Název WiFi zastřešuje množinu standardů IEEE 802.11 s různými příponami.
Nejznámější z nich jsou přípony: a, b, g, n. Další standardy jsou ve vývoji, nebo jejich nasazení není ještě běžné.
Tyto standardy se od sebe rozlišují zejména maximální přenosovou rychlostí a frekvencí pásma.
\begin{itemize}
    \item a -- přenosová rychlost: 54 Mb\,/\,s, pracovní frekvence: 5 GHz
    \item b -- přenosová rychlost: 11 Mb\,/\,s, pracovní frekvence: 2.4 GHz
    \item g -- přenosová rychlost: 54 Mb\,/\,s, pracovní frekvence: 2.4 GHz
    \item n -- přenosová rychlost: až 600 Mb\,/\,s, pracovní frekvence: 2.4 nebo 5 GHz
\end{itemize}
Výše uvedené rychlosti jsou pouze teoretické, v réalném nasazení jsou tyto rychlosti \cite{rychlost} mnohem nižší.
%
\section{Zabezpečení}
Zabezpečení WiFi nelze brát na lehkou váhu, proto byly s postupem času vyvinuty různé způsoby zabezpečení.
\begin{itemize}
    \item WEP -- jednoduché zabezpečení pomocí klíče složeného ze znaků hexadecimální soustavy
    \item WPA uživatelské -- dnes nejběžnější zabezpečení sítě v domácnostech pomocí klíče s 8 - 63 znaky
    \item WPA podnikové -- využívá k autentizaci uživatelů RADIUS \cite{metody_autentizace} protokol, bežné ve velkých firmách či školách
\end{itemize}
%
\subsection{WEP}
Zabezpečení WEP je jedno z prvních zabezpečení WiFi sítí. 
K šifrování komunikace pomocí tohoto algoritmu je použit klíč o délce 10 nebo 26 hexadecimálních znaků.
V případě délky 10 znaků se jedná o WEP~64-bit, v případě délky 26 znaků pak jako WEP~128-bit \cite{zabezp_kniha}.
Nekteří výrobci nabízeli zařízení schopné pracovat i s WEP~152-bit a také s WEP~256-bit \cite{wiki_wep}.\par
V dnešní době není již použití WEP bezpečné a toto zabezepčení lze prolomit \cite{wep_crack} prakticky během několika minut.
%
\subsection{WPA}
WPA -- neboli Wireless Protected Access je přímým nástupcem zabezpečení WEP. V dnešní době jde o relativně rozumné 
zabezpečení, avšak byly již objeveny způsoby, jak tento typ zabezpečení v rozumném čase prolomit \cite{wpa_crack}.
%
\subsubsection{WPA uživatelské}
Zabezpečení WPA \cite{wpa_personal} v uživatelské formě umožňuje zabezpečit probíhající komunikaci mezi zařízeními libovolným klíčem
o délce 8 -- 63 znaků. Tato forma zabezpečení WPA je běžně užívaná v domácnostech a malých firmách. Ve velkých podnicích
je vhodné využít podnikové WPA.
%
\subsubsection{WPA podnikové}
Nasazení podnikového WPA je možné pouze s RADIUS serverem, který zajišťuje jednak správu
uživatelů, jejich autorizaci a také autentizaci v síti. Tento způsob centrální správy umožňuje
vysokou míru zabezpečení a také umožňuje v rámci různých LAN jednotné ověřování a přihlašování. Typickým příkladem je
univerzitní síť eduroam \cite{eduroam}.
%
\section{WPA2}
V roce 2006 bylo zabezpečení WPA oficiálně nahrazeno zabezpečením WPA2. Na rozdíl od WPA, WPA2 vyžaduje
pro šifrování komunikace algoritmy AES, ale podporuje i algoritmus TKIP, aby byla zajištěna zpětná kompatibilita se zabezpečením WPA.
V roce 2010 byla ve WPA2 objevena \cite{hole196} chyba pojmenovaná \uv{Hole196} podle místa, kde byla nalezena -- v popisu 
protokolu IEEE 802.11 na straně 196. 
\newpage
\bibliography{zdroje}
%
\end{document}